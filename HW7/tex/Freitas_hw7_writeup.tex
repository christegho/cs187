\documentclass[12pt]{article}
\usepackage{qtree}
%Mathematical TeX packages from the AMS
\usepackage{amssymb,amsmath,amsthm} 
%geometry (sets margin) 
\usepackage[margin=1.25in]{geometry}
\usepackage{enumerate}
\usepackage{graphicx}

\def\ci{\perp\!\!\!\perp}
%=============================================================
%Redefining the 'section' environment as a 'problem' with dots at the end


\makeatletter
\newenvironment{problem x}{\@startsection
       {section}
       {1}
       {-.2em}
       {-3.5ex plus -1ex minus -.2ex}
       {2.3ex plus .2ex}
       {\pagebreak[3] %basic widow-orphan matching
       \large\bf\noindent{Problem }
       }
       }
       {%\vspace{1ex}\begin{center} \rule{0.3\linewidth}{.3pt}\end{center}}
       \begin{center}\large\bf \ldots\ldots\ldots\end{center}}
\makeatother


%=============================================================
%Fancy-header package to modify header/page numbering 
%
\usepackage{fancyhdr}
\pagestyle{fancy}
\usepackage{framed, color}
\definecolor{shadecolor}{gray}{0.95}
\lhead{Lucas Freitas}
\chead{} 
\rhead{\thepage} 
\lfoot{\small Computer Science 187} 
\cfoot{} 
\rfoot{\footnotesize Problem Set 7} 
\renewcommand{\headrulewidth}{.3pt} 
\renewcommand{\footrulewidth}{.3pt}
\setlength\voffset{-0.25in}
\setlength\textheight{648pt}
\setlength\parindent{0pt}
%=============================================================
%Contents of problem set

\begin{document}

\title{Problem Set 7: Syntactic Analysis\\\textsc{Computer Science 187}}
\author{Lucas Freitas}

\maketitle

\thispagestyle{empty}

% PROBLEM 1

\section*{1. Possessives}

  The problems below are from section $3.8.2$ of PNLA:

  \subsection*{Problem 3.15}
    To handle possessives, we only need to add the following line after the first instance of \texttt{det}:
    \begin{verbatim}
det(_, det(NP, [s])) --> np(_, NP), [s].
    \end{verbatim}
    If we try to test the Prolog program for possessives, however, we notice that we have a stack overflow. For instance: 
    \begin{verbatim}
?- s(A, B, [frog, s,  cookies], []).
ERROR: Out of local stack
    \end{verbatim}
    That doesn't mean that the program is incorrect. That just means that it is left-recursive, which makes it hard to test due and prone to stack overflows. We will correct that issue in the following items.

  \subsection*{Problem 3.16}
    We can eliminate left-recursion from the possessive grammar by rewriting it into one weakly equivalent to the original one. The way of doing that is transforming a relation of type $A \rightarrow A\beta | \alpha$ into two:

    $$A\rightarrow \alpha A'$$
    $$A' \rightarrow \beta A' | \epsilon$$

    \newpage
  \subsection*{Problem 3.16 (continued)}
    That can be done by doing:

    \begin{verbatim}
%% basic rules for NP
np(Agr, np(PN)) --> pn(Agr, PN).
np(Agr, np(Det, N)) --> det(Agr, Det), n(Agr, N).

%% NP with possessive (should agree with last element)
np(Agr, np(PN, Rest)) --> pn(_, PN), np_prime(Agr, Rest).
np(Agr, np(Det, N, Rest)) --> det(A, Det), n(A, N), np_prime(Agr, Rest).

%% possessive rules
np_prime(Agr, np([s], N, Rest)) --> [s], n(_, N), np_prime(Agr, Rest).
np_prime(Agr, np([s], N)) --> [s], n(Agr, N).
    \end{verbatim}

  \subsection*{Problem 3.17}

    The solution for $3.16$ already produces a parse tree as in the previous sections.

  \subsection*{Sample queries}

  Sample queries behave as expected:
  \begin{verbatim}
?- s(A,B,[toad,s,cookies,are,frog,s,cookies],[]).
A = agr(pl, third),
B = s(np(pn(toad), np([s], n(cookies))), vp(v(are),
cp(np(pn(frog), np([s], n(cookies)))))) .

?- s(A,B,[toad,s,cookies,s,cake,bakes,the,cookies],[]).
A = agr(sg, third),
B = s(np(pn(toad), np([s], n(cookies), np([s], n(cake)))),
vp(v(bakes), cp(np(det(the), n(cookies))))) .

?- s(A,B,[toad,s,cookies,s,cake],[]).
false.
  \end{verbatim}

  \newpage
  \section*{2. Prepositional Phrases}
    \subsection*{Problem 3.20}
      To add prepositional phrases, we first need to make adverbial prepositional phrases siblings of $S$ instead of $VP$, and nominal prepositional phrases, siblings of $NP$. Thus, all we need to add to our grammar is:

    \begin{verbatim}
%% Prepositional phrases
s(Agr, s(NP, VP, Prep)) --> np(Agr, NP), vp(Agr, VP,_,_), pp(Prep).
...
pp(pp(Prep, NP)) --> p(Prep), np(_, NP).
np(Agr, np(Det, N, Prep)) --> det(Agr, Det), n(Agr, N), pp(Prep).
...
p(p(X)) --> [X], {p(X)}.
p(for).     p(to).      p(in).
p(on).      p(above).   p(below).
p(with).    p(every).
    \end{verbatim}

    Notice how useful weakly equivalence is for helping determine what parse trees look like, and solve the left-recursion on the program.
    \subsection*{Sample queries}

    Sample queries behave as expected:
  \begin{verbatim}
?- s(A,B,[toad,bakes,the,cookies,in,the,kitchen],[]).
A = agr(sg, third),
B = s(np(pn(toad)), vp(v(bakes), cp(np(det(the), n(cookies),
pp(p(in), np(det(the), n(kitchen))))))) .

?- s(A,B,[every,cookie,in,toad,s,box,dies],[]).
A = agr(sg, third),
B = s(np(det(every), n(cookie), pp(p(in), np(pn(toad), np([s],
 n(box))))), vp(v(dies))) .

?- s(A,B,[toad,met,a,box,with,a,cake],[]).
A = agr(sg, third),
B = s(np(pn(toad)), vp(v(met), cp(np(det(a), n(box), pp(p(with),
 np(det(a), n(cake))))))) .

?- s(A,B,[toad,baked,a,cake,for,a,cookie],[]).
A = agr(sg, third),
B = s(np(pn(toad)), vp(v(baked), cp(np(det(a), n(cake), 
pp(p(for), np(det(a), n(cookie))))))) .
  \end{verbatim}

  \newpage
  \section*{3. Verb Subcategorization}
    \subsection*{Problem 3.23}

    Following the guidelines on PNLA, we can add verb subcategorization to our grammar by adding an argument, \texttt{Type}, to represent the transitivity of the verb. We can also create a new component for a grammar: verbal complements (\texttt{cp}). Using that method, all that we need to do is:

    \begin{verbatim}
%% Verbal phrases

vp(Agr, vp(V), intransitive, Conj) -->
               v(Agr, V, intransitive, Conj).
vp(Agr, vp(V, Comps), Type, Conj) -->
               v(Agr, V, Type,Conj), cp(Type, Comps).

%% Verbal complements

cp(transitive, cp(NP)) --> np(_, NP).
cp(ditransitive, cp(NP1, NP2)) --> np(_, NP1), np(_, NP2).
cp(dative(Prep), cp(NP1, pp(pp(p(Prep), NP2)))) -->
               np(_, NP1), pp(pp(p(Prep), NP2)).
...

%% Verbs

v(agr(sg,first), v(X), Type, Conj) --> [X], {vlex(X,_,_,_,Type,Conj)}.
v(agr(sg,second), v(X), Type, Conj) --> [X], {vlex(_,X,_,_,Type,Conj)}.
v(agr(sg,third), v(X), Type, Conj) --> [X], {vlex(_,_,X,_,Type,Conj)}.
v(agr(pl,_), v(X), Type, Conj) --> [X], {vlex(_,_,_,X,Type,Conj)}.

% to meet
vlex(meet, meet, meets, meet, transitive).

% to be
vlex(am, are, is, are, transitive).

% to bake
vlex(bake, bake, bakes, bake, ditransitive).
vlex(bake, bake, bakes, bake, transitive).
vlex(bake, bake, bakes, bake, dative(for)).
vlex(bake, bake, bakes, bake, dative(in)).

% to put
vlex(put, put, puts, put, dative(in)).

% to die
vlex(die, die, dies, die, intransitive).

% to give
vlex(give, give, gives, give, ditransitive).
vlex(give, give, gives, give, dative(to)).
    \end{verbatim}
  \subsection*{Sample queries}

    Sample queries behave as expected:
  \begin{verbatim}

?- s(A,B,[toad,bakes,the,cookies],[]).
A = agr(sg, third),
B = s(np(pn(toad)), vp(v(bakes), cp(np(det(the), n(cookies))))) .

?- s(A,B,[toad,bakes,frog,the,cookies],[]).
A = agr(sg, third),
B = s(np(pn(toad)), vp(v(bakes), cp(np(pn(frog)), np(det(the), 
n(cookies))))) .

?- s(A,B,[toad,bakes,the,cookies,to,frog],[]).
A = agr(sg, third),
B = s(np(pn(toad)), vp(v(bakes), cp(np(det(the), n(cookies), 
pp(p(to), np(pn(frog))))))) .

?- s(A,B,[toad,bakes,the,cookies,to,frog,in,the,kitchen],[]).
A = agr(sg, third),
B = s(np(pn(toad)), vp(v(bakes), cp(np(det(the), n(cookies), 
pp(p(to), np(pn(frog)))), pp(pp(p(in), np(det(the), 
n(kitchen))))))) .

?- s(A,B,[toad,dies],[]).
A = agr(sg, third),
B = s(np(pn(toad)), vp(v(dies))) .
  \end{verbatim}

  \newpage
  \section*{4. Auxiliary Verbs}
    \subsection*{Problem 4.2.1}

    Using a similar approach to Problem $3$, we can add another argument for the conjugation of the verb, \texttt{Conj}, and do a similar approach to the one suggested in PNLA. The declaration of all verbe tenses was supressed from this writeup, but can be found at \texttt{frog-toad-3a.pl}.

    \subsection*{Prolog code}
    \begin{verbatim}
%% Auxiliar verbs

vp(Agr, vp(Aux, V), intransitive, Conj) --> 
        aux(Aux, Conj / Req), vp(Agr, V, intransitive, Req).
vp(Agr, vp(Aux, Rest), Type, Conj) -->
        aux(Aux, Conj / Req), vp(Agr, Rest, Type, Req).
...
aux(aux(X), Form) --> [X], {aux(X,Form)}.

% can
aux(can, none / nonfinite).

% could
aux(could, finite / nonfinite).

% have
aux(have, nonfinite / past).

% been
aux(been, past / present).

% be
aux(be, nonfinite / present).
    \end{verbatim}

  \newpage
  \section*{5. Sample queries}

  All works as expected.
      \begin{verbatim}
?- s(Agr,S,[toad,gave,some,cookies,to,frog],[]).
Agr = agr(sg, third),
S = s(np(pn(toad)), vp(v(gave), cp(np(det(some), n(cookies)), 
pp(pp(p(to), np(pn(frog))))))) .

?- s(Agr,S,[toad,gave,frog,some,cookies],[]).
Agr = agr(sg, third),
S = s(np(pn(toad)), vp(v(gave), cp(np(pn(frog)), np(det(some), 
n(cookies))))) .

?- s(Agr,S,[frog,put,the,cookies,in,the,box],[]).
Agr = agr(sg, third),
S = s(np(pn(frog)), vp(v(put), cp(np(det(the), n(cookies)), 
pp(pp(p(in), np(det(the), n(box))))))) .

?- s(Agr,S,[frog,put,the,box,some,cookies],[]).
false.

?- s(Agr,S,[frog,put,the,cookies],[]).
false.

?- s(Agr,S,[toad,baked,some,cookies,for,frog],[]).
Agr = agr(sg, third),
S = s(np(pn(toad)), vp(v(baked), cp(np(det(some), n(cookies), 
pp(p(for), np(pn(frog))))))) .

?- s(Agr,S,[toad,baked,some,cookies,in,the,kitchen],[]).
Agr = agr(sg, third),
S = s(np(pn(toad)), vp(v(baked), cp(np(det(some), n(cookies), 
pp(p(in), np(det(the), n(kitchen))))))) .

?- s(Agr,S,[toad,gave,frog,some,cookies,in,the,kitchen],[]).
Agr = agr(sg, third),
S = s(np(pn(toad)), vp(v(gave), cp(np(pn(frog)), np(det(some), 
n(cookies), pp(p(in), np(det(the), n(kitchen))))))) .

?- s(Agr,S,[toad,can,bake,a,cake],[]).
Agr = agr(sg, third),
S = s(np(pn(toad)), vp(aux(can), vp(v(bake), cp(np(det(a), 
n(cake)))))) .

?- s(Agr,S,[toad,could,have,been,baking,a,cake],[]).
Agr = agr(sg, third),
S = s(np(pn(toad)), vp(aux(could), vp(aux(have), vp(aux(been), 
vp(v(baking), cp(np(det(a), n(cake)))))))) .

?- s(A,B,[toad,could,have,bake,a,cake],[]).
false.
      \end{verbatim}


\end{document}